\documentclass[10pt,a4paper]{report}
\usepackage[latin1]{inputenc}
\author{Hildeberto Mendon\c{c}a}
\title{JUG Management Application}
\begin{document}
\maketitle
\tableofcontents
\part{User Guide}
\chapter{Introduction}

\section{The Art of Managing a Java User Group}

* Can you tell us about the application, site, or service in which you have
  adopted GlassFish?
   [ Note: this is where you can hopefully get some publicity for your
     own business or project.  So consider including any hyperlinks,
     screenshots, etc. that you would like for us to use in that context. ]

\section{An Application to Manage JUGs}

The JUG Management application explicitly aims to increase and strengthen the Java Community by promoting control of JUG operations and managing the knowledge produced by members of the group. We believe that knowledge and opportunities are the main forces of attraction around Java User Groups and an application is needed to make these forces efficiently flow through the community. It is also important to emphasize that JUG leaders have always been great entrepreneurs, but they lack of analytical information to make mature decisions.

The CEJUG Community (http://www.cejug.org) is leading the project and decided to make it widely available as an open source software. It is freely available for use and open for contributions. CEJUG is using the first release of the application (http://www.cejug.org:8080/cejug) since January 1st, 2011. It is helping the group to define what is actually being a JUG member. Nowadays, most of JUGs simply consider all those people registered in their technical mailing as members. This simplicity is good for management purposes, but we lose lots of information because of that. We don't know, for instance, which reasons led members to leave the group. Did we do something wrong? What can we do to get better and have members back into the boat? We also noticed that even non-technical people, as entrepreneurs, recruiters, and those who decided to unsubscribe because of too many messages, would like to keep in touch with the group, not necessarily going into technical discussions, but proposing other ways of collaboration. Adopting a separate application to manage subscriptions is helping CEJUG to collect more feedback and be more inclusive.

\chapter{Membership Management}

\section{Registration}

The decision to register in the user group always come from the interested person. That is why the only way to add a new member is filling out the initial registration form, accessible through a link in the application header when there is no member logged in. JUG leaders do not have any feature that allow them to add new members manually.

When the interested person submits the registration form, we have to make sure that his/her email address is correct before considering him/her as a member. We send an email message to the email address informed in the form, asking the interested person to confirm their email address by clicking on the confirmation link. This link contains a unique code that guarantees that the link cannot be reused after its first use, confirming the email address only once. The interested person is considered as a member as soon as his/her email address is confirmed. The new member receives a welcome email message and JUG Leaders are informed by email about the successful member registration.

\section{Member Profile}

The member has the right to read and modify any data published on its profile. The email address is the only data subject of validation. The email validation works the same way it works during the registration, sending a message to the new email address with a link to confirm it.

\section{Deactivation}

The deactivation of a member means that he/she will not participate in the activities of the group  starting from the date of the deactivation. No email message, invitation, offer, or any other kind of information will be sent to the deactivated member anymore. Therefore, he/she is not considered as a regular member.

At the same time, all data inputed by the ancient member in the database will not be removed. Comments, email messages, articles and other information will be kept unchanged indefinitely. Therefore, any modification on these data is not responsibility of the application.

\chapter{Event Management}

Events are strategic for user groups. They disseminate knowledge and promote networking, strengthening the links between members. It is important to have an efficient event management in order to keep everything under control, measure member's participation and get their feedback.

To start organizing events, UG Leaders should register venues available and suitable for user group events. A venue might have one or more rooms where event sessions will take place. In case of multiple rooms available, the event can manage multiple sessions in parallel.

Events are allocated in existing venues for a certain period of time. When an event is registered and allocated to a venue, the venue's contact receives an request email containing details about the event and a list of resources that are expected from them. After a negotiation process the event is confirmed or not. The confirmation occurs when the venue's contact clicks on the confirmation link, at the end of the email message. Without this confirmation the event cannot occur.

When confirmed, an email message containing detailed information about the event is sent to all members that have declared in their registration form the wish to receive information about events. This message contains a direct link to the event registration form.

Detailed information about the event is also formatted to be published on the UG website. Consequently, it may attract people who are interested on the event but are not member of the UG. Those people should become a member of the UG before registering to the event.

Right after the event registration, the member receives a confirmation message just to let him/her know that he/she is successfully registered to attend the event. Registered members will receive a remind email message seven days before the event and a second one on the day before the event. They can cancel their registration at any time before the event.

At the entrance of the event, a member of the UG staff checks the inscription of each person in an available computer. If the person is a member registered in the event, then his/her presence is confirmed. If the person is a member but he/she is not registered in the event, then his/her registration is made at the entrance. If the person is not a member, then he/she should agree to become a member of the UG, otherwise it is not possible to join the event. If he/she agrees to become a member, his/her registration in the UG and in the event is done at the entrance of the event. 

\part{Developer Guide}
\chapter{Software Architecture}

\section{Application Server}
The first deployment of the JUG Management application was made in the Glassfish V3 Application Server. 

* How and when did you first find out about GlassFish?

Project leaders have been in contact with Glassfish since its early versions. Other CEJUG projects, such as the PUG Arena and CEJUG Classifieds, were developed and deployed on Glassfish V2.

* Did you go through an evaluation process before selecting GlassFish?

The choice for Glassfish was based on previous experiences, expertise of the team and our current hosting possibilities. No comparison was made with other application servers because, at the time, it was not possible, since Glassfish was the only application server available implementing the latest JEE specification.

* What specific version of GlassFish are you using?

We are using Glassfish V3.0.1 Open Source Edition

* On what operating system do you run GlassFish?  Do you use the same OS for
  both development and production deployment?
  
We are using Ubuntu and Windows in the development environment and CentOS in the production environment. However, since it is a open source project, the application must run in all Java-enabled operating systems.

* On what hardware platform do you run GlassFish?  Do you use the same platform
  for both development and production deployment?
  
  The hardware on production environment is actually provided by the hosting service. As far as we know, it is a Linux virtual machine with limited memory and processing capability. For the moment, we are able to support only a few concurrent users, but it is due to resources limitation, not application constraints.

* Are you a paying customer? If not, do you know what supported customers are entitled to?

No. We are not. No, we don't know.

* What specific features or modules of GlassFish are you using?

We decided to adopt an minimalist approach where most of libs needed by the application are also distributed with Glassfish, such as Eclipse Link and Mojarra, and most of configurations are made through the administrative console, such as database connection pool, JavaMail session and Security Realm. The only external lib is the Primefaces component library. We would love to find a way to avoid that too, but not possible :) We make extensive use of annotations and avoid as much as we can xml for configuration. The transaction is fully managed by the container. This way, we keep focused on the source code of the JUG community model.

* What do you like most about GlassFish?

Its administrative console is awesome and its hot deployment with Netbeans is just perfect.

* What would you most like to see improved in GlassFish?

As we said before, all we would love to have for the moment is an official visual component library available by default. This is the only lib missing for a fully supported JEE application, such as the JUG Management one. We know this is a difficult decision, but we would support it.

* Are you using any open source or commercial frameworks or tools in your application?
  (Examples could include Seam, Spring, Hibernate, ...)
  
  Out of Glassfish, only Primefaces.

* Does your application use a database? If so, which one? 

We use MySQL.

* Are there any figures about the scale of your adoption which you would like
  to share (such as how much traffic is being handled, how many servers are
  used, how much developer time went into building your application)?
  
About that, we don't have much to say, due to the size of the application and the limitation of budget.  

* How has GlassFish performed since your application went live?  Have you run
  into any production issues which you would attribute to GlassFish?
  
It is running like a charm ;) No problem so far. Let's see how many deployments it can resist before any possible memory leak.  

* How would your describe your participation in the GlassFish project (e.g.
  user only, submitter of bug reports and RFEs, developer who has contributed
  code)?
  
  We are one users so far. We would love to contribute more.

* Is there anything else you think would be of interest in a story about your
  GlassFish adoption? 
  
This JEE Server has gained lots of attention for constantly run towards the state-of-the-art of the Java server side technology. It has an aggressive roadmap, being always the first product on the market to fully implement the last version of the JEE specification. This commitment with the Java technology led CEJUG to adopt it in order to maximize our productivity.
  
\end{document}